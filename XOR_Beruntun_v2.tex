\documentclass{article}

\usepackage{geometry}
\usepackage{amsmath}
\usepackage{graphicx}
\usepackage{listings}
\usepackage{hyperref}
\usepackage{multicol}
\usepackage{fancyhdr}
\pagestyle{fancy}
\hypersetup{ colorlinks=true, linkcolor=black, filecolor=magenta, urlcolor=cyan}
\geometry{ a4paper, total={170mm,257mm}, top=20mm, right=20mm, bottom=20mm, left=20mm}
\setlength{\parindent}{0pt}
\setlength{\parskip}{1em}
\renewcommand{\headrulewidth}{0pt}
\lhead{Competitive Programming - Arkavidia V}
\fancyfoot[CE,CO]{\thepage}
\lstset{
    basicstyle=\ttfamily\small,
    columns=fixed,
    extendedchars=true,
    breaklines=true,
    tabsize=2,
    prebreak=\raisebox{0ex}[0ex][0ex]{\ensuremath{\hookleftarrow}},
    frame=none,
    showtabs=false,
    showspaces=false,
    showstringspaces=false,
    prebreak={},
    keywordstyle=\color[rgb]{0.627,0.126,0.941},
    commentstyle=\color[rgb]{0.133,0.545,0.133},
    stringstyle=\color[rgb]{01,0,0},
    captionpos=t,
    escapeinside={(\%}{\%)}
}

\begin{document}

\begin{center}
    \section*{XOR Beruntun} % ganti judul soal

    \begin{tabular}{ | c c | }
        \hline
        Batas Waktu  & 1s \\    % jangan lupa ganti time limit
        Batas Memori & 32MB \\  % jangan lupa ganti memory limit
        \hline
    \end{tabular}
\end{center}

\subsection*{Deskripsi}

Suatu saat pada sore hari, Elza bersama ($1 \leq N \leq 2.10^5$) temannya melaksanakan arisan bulanan.
Karena bosan dengan pengundian menggunakan kertas, Elza bersama temannya membuat sebuah pseudo random generator.
Pseudo random generator bekerja dengan cara memasukkan banyaknya peserta arisan lalu peserta arisan bergantian memasukkan sebuah angka $X$ yang tidak kurang dari 0 dan tidak lebih dari $10^9$.
Selanjutnya, angka yang dimasukkan peserta arisan pertama akan dilakukan proses $XOR$ dengan angka peserta arisan kedua, 
angka yang dimasukkan peserta arisan kedua akan dilakukan proses $XOR$ dengan peserta arisan ketiga, 
dan begitu seterusnya hingga angka peserta arisan ke-$N - 1$ dan ke-$N$ diproses.
Pemrosesan tersebut dilakukan terus menerus hingga didapatkan satu angka saja.
Elza sebagai pemenang arisan sebelumnya tidak dapat mengikuti arisan dan hanya menjadi operator pseudo random generator.

\subsection*{Format Masukan}

Pada baris pertama dimasukkan $N$.
Pada baris kedua dimasukkan array $A$ dengan panjang $N$.

\subsection*{Format Keluaran}

Keluarkan sebuah integer $Y$ hasil dari pseudo random generator.
\\

\begin{multicols}{2}
\subsection*{Contoh Masukan}
\begin{lstlisting}
3
1 3 5
\end{lstlisting}
\columnbreak
\subsection*{Contoh Keluaran}
\begin{lstlisting}
4
\end{lstlisting}
\vfill
\null
\end{multicols}

% \subsection*{Penjelasan}
% Jika dibutuhkan, tambahkan penjelasan di sini

\pagebreak

\end{document}

