\documentclass{article}

\usepackage{geometry}
\usepackage{amsmath}
\usepackage{graphicx}
\usepackage{listings}
\usepackage{hyperref}
\usepackage{multicol}
\usepackage{fancyhdr}
\pagestyle{fancy}
\hypersetup{ colorlinks=true, linkcolor=black, filecolor=magenta, urlcolor=cyan}
\geometry{ a4paper, total={170mm,257mm}, top=20mm, right=20mm, bottom=20mm, left=20mm}
\setlength{\parindent}{0pt}
\setlength{\parskip}{1em}
\renewcommand{\headrulewidth}{0pt}
\lhead{Competitive Programming - Arkavidia VI}
\fancyfoot[CE,CO]{\thepage}
\lstset{
    basicstyle=\ttfamily\small,
    columns=fixed,
    extendedchars=true,
    breaklines=true,
    tabsize=2,
    prebreak=\raisebox{0ex}[0ex][0ex]{\ensuremath{\hookleftarrow}},
    frame=none,
    showtabs=false,
    showspaces=false,
    showstringspaces=false,
    prebreak={},
    keywordstyle=\color[rgb]{0.627,0.126,0.941},
    commentstyle=\color[rgb]{0.133,0.545,0.133},
    stringstyle=\color[rgb]{01,0,0},
    captionpos=t,
    escapeinside={(\%}{\%)}
}

\begin{document}

\begin{center}
    \section*{Expected Value} % ganti judul soal

    \begin{tabular}{ | c c | }
        \hline
        Batas Waktu  & 1s \\    % jangan lupa ganti time limit
        Batas Memori & 64MB \\  % jangan lupa ganti memory limit
        \hline
    \end{tabular}
\end{center}

\subsection*{Deskripsi}
Saat berada di dalam hutan misterius, Elza dan Anda bertemu dengan suku matahari yang merupakan suku asli di dalam hutan tersebut.
Kalian disambut dengan lapang dada oleh mereka. Mereka pun membuat sebuah pesta makan bersama untuk menyambut kalian.
Pada saat menikmati makanan, Elza melihat anak-anak suku matahari memainkan sebuah game unik. Game ini bernama \textit{Expected Value}. Elza pun
mengajak Anda untuk memainkannya.

Diberikan sebuah bilangan $N$. Kalian diminta untuk memilih sebuah bilangan acak yang berada di antara angka $0$ sampai $N-1$ yang 
direpresentasikan dengan biner.Panjangnya biner setiap angka akan sama panjang mengikuti bilangan terbesar, yaitu $N-1$.
Cara memilihnya yaitu dengan melemparkan sebuah koin dengan 2 sisi, sisi yang bertuliskan $0$ dan $1$.

Setelah mendengar penjelasan permainnya, Elza memberikan sebuah tantangan kepada Anda. Anda diminta untuk menghitung berapa 
ekspektasi banyak pelemparan koin untuk mendapatkan bilangan yang valid ($0 < B < N-1$). Dapatkah Anda menaklukkan tantangan dari Elza?

\subsection*{Format Masukan}
Baris pertama terdiri dari satu bilangan bulat positif $N$ ($1 \leq N \leq 1e18$).

\subsection*{Format Keluaran}
Sebuah baris yang menyatakan besar ekspektasi pelemparan koin untuk mendapatkan bilangan diantara $[0..N-1]$
\\

\begin{multicols}{2}
\subsection*{Contoh Masukan}
\begin{lstlisting}
1
\end{lstlisting}
\columnbreak
\subsection*{Contoh Keluaran}
\begin{lstlisting}
2
\end{lstlisting}
\vfill
\null
\end{multicols}


\pagebreak

\end{document}

