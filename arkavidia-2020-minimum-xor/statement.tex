\documentclass{article}

\usepackage{geometry}
\usepackage{amsmath}
\usepackage{graphicx}
\usepackage{listings}
\usepackage{hyperref}
\usepackage{multicol}
\usepackage{fancyhdr}
\pagestyle{fancy}
\hypersetup{ colorlinks=true, linkcolor=black, filecolor=magenta, urlcolor=cyan}
\geometry{ a4paper, total={170mm,257mm}, top=20mm, right=20mm, bottom=20mm, left=20mm}
\setlength{\parindent}{0pt}
\setlength{\parskip}{1em}
\renewcommand{\headrulewidth}{0pt}
\lhead{Competitive Programming - Arkavidia VI}
\fancyfoot[CE,CO]{\thepage}
\lstset{
    basicstyle=\ttfamily\small,
    columns=fixed,
    extendedchars=true,
    breaklines=true,
    tabsize=2,
    prebreak=\raisebox{0ex}[0ex][0ex]{\ensuremath{\hookleftarrow}},
    frame=none,
    showtabs=false,
    showspaces=false,
    showstringspaces=false,
    prebreak={},
    keywordstyle=\color[rgb]{0.627,0.126,0.941},
    commentstyle=\color[rgb]{0.133,0.545,0.133},
    stringstyle=\color[rgb]{01,0,0},
    captionpos=t,
    escapeinside={(\%}{\%)}
}

\begin{document}

\begin{center}
    \section*{A. Minimum XOR} % ganti judul soal

    \begin{tabular}{ | c c | }
        \hline
        Batas Waktu  & 1s \\    % jangan lupa ganti time limit
        Batas Memori & 64MB \\  % jangan lupa ganti memory limit
        \hline
    \end{tabular}
\end{center}

\subsection*{Deskripsi}

XOR adalah salah satu operator dalam aljabar boolean. Operator ini menerima dua buah operan berupa nilai boolean ($true, false$ 
atau $0, 1$) dan menghasilkan $true$ atau $1$ jika salah satu operan bernilai $true$ atau $1$ tetapi tidak keduanya. 
Tabel kebenaran untuk operator ini adalah sebagai berikut.

\begin{center}
\begin{tabular}{c|c|c}
A & B & A XOR B \\ \hline
0 & 0 & 0       \\
1 & 0 & 1       \\
0 & 1 & 1       \\
1 & 1 & 0
\end{tabular}
\end{center}
Operasi XOR dapat dilakukan dengan operan berupa bilangan dalam basis biner,
 dengan cara melakukan operasi XOR pada tiap digit kedua operan yang bersesuaian.

Contoh:
\begin{center}
$10110101_2$ xor $11011001_2 = 01101100_2$
\end{center}

Sebuah bilangan bulat non-negatif $N$ dapat dihasilkan dengan menjumlahkan $k$ buah biangan bulat non-negatif ($a_1, a_2, a_3, ... a_k$), 
Dalam persoalan ini, kamu diminta mencari Minimum XOR, yaitu nilai minimum yang dapat dihasilkan dari operasi  $a_1$ xor $a_2$ xor $a_3$ xor $...$ xor $a_k$.

\subsection*{Format Masukan}

Masukan terdiri dari dua bilangan  $N, K$ ($1 \leq N \leq 10.000.000, 1 \leq K \leq 100.000$), dengan $N = a_1 + a_2 + a_3 + ... + a_K$.

\subsection*{Format Keluaran}

Keluaran berupa nilai minimum yang mungkin dihasilkan oleh $a_1$ xor $a_2$ xor $a_3$ xor $...$ xor $a_K$
\\

\begin{multicols}{2}
\subsection*{Contoh Masukan}
\begin{lstlisting}
10 4
\end{lstlisting}
\columnbreak
\subsection*{Contoh Keluaran}
\begin{lstlisting}
0
\end{lstlisting}
\vfill
\null
\end{multicols}

\subsection*{Penjelasan}
Empat bilangan dapat dijumlahkan menjadi 10
\[ 3 + 3 + 2 + 2 = 10 \]
Keempat bilangan ini dapat di-XOR-kan menghasilkan nilai minimum
\begin{center}
$11_2$ xor $11_2$ xor $10_2$ xor $10_2 = 0$
\end{center}
\pagebreak

\end{document}
