\documentclass{article}

\usepackage{geometry}
\usepackage{amsmath}
\usepackage{graphicx}
\usepackage{listings}
\usepackage{hyperref}
\usepackage{multicol}
\usepackage{fancyhdr}
\pagestyle{fancy}
\hypersetup{ colorlinks=true, linkcolor=black, filecolor=magenta, urlcolor=cyan}
\geometry{ a4paper, total={170mm,257mm}, top=20mm, right=20mm, bottom=20mm, left=20mm}
\setlength{\parindent}{0pt}
\setlength{\parskip}{1em}
\renewcommand{\headrulewidth}{0pt}
\lhead{Competitive Programming - Arkavidia VI}
\fancyfoot[CE,CO]{\thepage}
\lstset{
    basicstyle=\ttfamily\small,
    columns=fixed,
    extendedchars=true,
    breaklines=true,
    tabsize=2,
    prebreak=\raisebox{0ex}[0ex][0ex]{\ensuremath{\hookleftarrow}},
    frame=none,
    showtabs=false,
    showspaces=false,
    showstringspaces=false,
    prebreak={},
    keywordstyle=\color[rgb]{0.627,0.126,0.941},
    commentstyle=\color[rgb]{0.133,0.545,0.133},
    stringstyle=\color[rgb]{01,0,0},
    captionpos=t,
    escapeinside={(\%}{\%)}
}

\begin{document}

\begin{center}
    \section*{Loncat Kuda} % ganti judul soal

    \begin{tabular}{ | c c | }
        \hline
        Batas Waktu  & 1s \\    % jangan lupa ganti time limit
        Batas Memori & 128MB \\  % jangan lupa ganti memory limit
        \hline
    \end{tabular}
\end{center}

\subsection*{Deskripsi}

Arvy sedang memainkan permainan yang baru saja dia buat, dinamakan loncat kuda. 
Aturannya sederhana, sebuah kuda catur harus digerakkan menuju sebuah titik akhir pada papan catur $N \times M$.
Pada permainan ini, kuda dapat meloncat sejauh $A$ langkah vertikal dan $B$ langkah horizontal atau $A$ langkah horizontal dan $B$ langkah vertikal.
Selain itu, terdapat sejumlah jebakan pada papan catur yang tidak boleh dilewati kuda.

Berikut adalah contoh papan permainan.

\begin{center}
    \includegraphics[width=100px]{final/Initial-Configuration.png}    
\end{center}


Misi Anda adalah membantu Arvy agar dapat menyelesaikan permainan dalam langkah terkecil.
Misalnya, untuk $(A,B)=(3,1)$, konfigurasi diatas dapat diselesaikan dalam langkah-langkah berikut.

\begin{center}
    \includegraphics[width=100px]{Movelist}
\end{center}

Dapatkah Anda membantu Arvy?

\subsection*{Format Masukan}

Baris pertama terdiri dari dua bilangan positif $N$ dan $M$ ($1 \leq N \leq 100.000$, $1 \leq M \leq 100.000$, $1 \leq N \times M \leq 100.000$) yang menyatakan ukuran baris dan kolom papan permainan.
$N$ baris selanjutnya diisi dengan masing-masing $M$ karakter yang merupakan salah satu dari: X, -, K, O.

\begin{itemize}
    \setlength\itemsep{0pt}
    \item Karakter X merupakan sebuah perangkap sehingga tidak bisa dilewati,
    \item Karakter - berarti petak tidak ditempati hal apapun dan dapat dilewati,
    \item Karakter K menunjukkan adanya kuda pada petak tersebut, dan
    \item Karakter O adalah titik tujuan kuda catur.
\end{itemize}

Baris terakhir terdiri dari 2 bilangan positif $A$ dan $B$ ($1 \leq A \leq 100.000$, $1 \leq B \leq 100.000$) yaitu ukuran langkah secara vertikal dan horizontal yang dapat dilakukan oleh kuda.

\subsection*{Format Keluaran}

Sebuah bilangan bulat positif yang menyatakan  langkah minimum yang harus dilakukan agar kuda dapat mencapai titik tujuannya.
\\

\begin{multicols}{2}
\subsection*{Contoh Masukan}
\begin{lstlisting}
6 5
X----
--X--
-K---
O-X--
----X
-----
3 1
\end{lstlisting}
\columnbreak
\subsection*{Contoh Keluaran}
\begin{lstlisting}
3
\end{lstlisting}
\vfill
\null
\end{multicols}

% \subsection*{Penjelasan}
% Jika dibutuhkan, tambahkan penjelasan di sini

\pagebreak

\end{document}

