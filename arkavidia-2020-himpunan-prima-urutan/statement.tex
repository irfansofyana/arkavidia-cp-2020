\documentclass{article}

\usepackage{geometry}
\usepackage{amsmath}
\usepackage{graphicx}
\usepackage{listings}
\usepackage{hyperref}
\usepackage{multicol}
\usepackage{fancyhdr}
\pagestyle{fancy}
\hypersetup{ colorlinks=true, linkcolor=black, filecolor=magenta, urlcolor=cyan}
\geometry{ a4paper, total={170mm,257mm}, top=20mm, right=20mm, bottom=20mm, left=20mm}
\setlength{\parindent}{0pt}
\setlength{\parskip}{1em}
\renewcommand{\headrulewidth}{0pt}
\lhead{Competitive Programming - Arkavidia VI}
\fancyfoot[CE,CO]{\thepage}
\lstset{
    basicstyle=\ttfamily\small,
    columns=fixed,
    extendedchars=true,
    breaklines=true,
    tabsize=2,
    prebreak=\raisebox{0ex}[0ex][0ex]{\ensuremath{\hookleftarrow}},
    frame=none,
    showtabs=false,
    showspaces=false,
    showstringspaces=false,
    prebreak={},
    keywordstyle=\color[rgb]{0.627,0.126,0.941},
    commentstyle=\color[rgb]{0.133,0.545,0.133},
    stringstyle=\color[rgb]{01,0,0},
    captionpos=t,
    escapeinside={(\%}{\%)}
}

\begin{document}

\begin{center}
    \section*{Himpunan Bilangan Prima} % ganti judul soal

    \begin{tabular}{ | c c | }
        \hline
        Batas Waktu  & 1s \\    % jangan lupa ganti time limit
        Batas Memori & 64MB \\  % jangan lupa ganti memory limit
        \hline
    \end{tabular}
\end{center}

\subsection*{Deskripsi}

Elza adalah orang yang sangat menyukai matematika. Saat ini ia sedang terfokus pada teori bilangan tepatnya 
sedang mempelajari faktor bilangan dan bilangan prima. Elza juga orang yang senang membuat soal-soal matematika, 
oleh sebab itu saat ini ia memikirkan sebuah soal yang dapat berhubungan dengan bilangan prima dan faktor bilangan. 
Setelah beberapa saat, Elza akhirnya mempunyai soal sebagai berikut.

Diberikan sebuah himpunan $H$ yang berisi himpunan bilangan prima. Dari himpunan tersebut, akan dibangkitkan 
semua bilangan bulat positif yang mana faktor prima-nya adalah \textit{subset} dari himpunan $H$ tersebut. 
Bilangan-bilangan yang dibangkitkan tersebut kemudian akan diurutkan dari terkecil hingga terbesar. Bilangan 
terkecil akan diberikan urutan-1. Apabila $K$ adalah bilangan bulat positif yang faktor prima-nya \textit{subset}
dari himpunan $H$, maka ada di urutan keberapakah bilangan $K$ tersebut?

Anda yang merupakan peserta Arkavidia diminta untuk menyelesaikan persoalan Elza. Dapatkah Anda menyelesaikannya?

\subsection*{Format Masukan}
Baris pertama berisi sebuah bilangan bulat positif $N$ ($1 \leq N \leq 100$) yaitu banyaknya bilangan pada
himpunan $H$. Baris berikutnya berisi $N$ buah bilangan $H_i$ ($1 \leq H_i \leq 2.000.000$), yaitu elemen himpunan $H$. $H_i$ dipastikan
bilangan prima. Baris terakhir berisi sebuah bilangan bulat positif $K$ ($1 \leq K \leq 10^{18}$). Faktor prima dari $K$ dipastikan
\textit{subset} dari himpunan $H$.

\subsection*{Format Keluaran}

Keluarkan satu bilangan bulat yang menyatakan urutan bilangan $K$ sesuai persoalan Elza.
\\

\begin{multicols}{2}
\subsection*{Contoh Masukan}
\begin{lstlisting}
2
2 3
9
\end{lstlisting}
\columnbreak
\subsection*{Contoh Keluaran}
\begin{lstlisting}
6
\end{lstlisting}
\vfill
\null
\end{multicols}

% \subsection*{Penjelasan}
% Jika dibutuhkan, tambahkan penjelasan di sini

\subsection*{Penjelasan}
Pada masukan di atas, H $ = \{2, 3\}$, sehingga bilangan-bilangan yang akan dibangkitkan adalah 
$\{2, 3, 4, 6, 8, 9, 12, ... \}$
Bilangan 9 menempati urutan ke-6.

\pagebreak

\end{document}

