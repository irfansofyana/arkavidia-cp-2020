\documentclass{article}

\usepackage{geometry}
\usepackage{amsmath}
\usepackage{graphicx}
\usepackage{listings}
\usepackage{hyperref}
\usepackage{multicol}
\usepackage{fancyhdr}
\pagestyle{fancy}
\hypersetup{ colorlinks=true, linkcolor=black, filecolor=magenta, urlcolor=cyan}
\geometry{ a4paper, total={170mm,257mm}, top=20mm, right=20mm, bottom=20mm, left=20mm}
\setlength{\parindent}{0pt}
\setlength{\parskip}{1em}
\renewcommand{\headrulewidth}{0pt}
\lhead{Competitive Programming - Arkavidia VI}
\fancyfoot[CE,CO]{\thepage}
\lstset{
    basicstyle=\ttfamily\small,
    columns=fixed,
    extendedchars=true,
    breaklines=true,
    tabsize=2,
    prebreak=\raisebox{0ex}[0ex][0ex]{\ensuremath{\hookleftarrow}},
    frame=none,
    showtabs=false,
    showspaces=false,
    showstringspaces=false,
    prebreak={},
    keywordstyle=\color[rgb]{0.627,0.126,0.941},
    commentstyle=\color[rgb]{0.133,0.545,0.133},
    stringstyle=\color[rgb]{01,0,0},
    captionpos=t,
    escapeinside={(\%}{\%)}
}

\begin{document}

\begin{center}
    \section*{Infinity-String} % ganti judul soal

    \begin{tabular}{ | c c | }
        \hline
        Batas Waktu  & 1s \\    % jangan lupa ganti time limit
        Batas Memori & 64MB \\  % jangan lupa ganti memory limit
        \hline
    \end{tabular}
\end{center}

\subsection*{Deskripsi}

Setelah Anda menyelesaikan Final \textit{competitive programming} Arkavidia 2020, Anda berkeliling ITB. Saat tengah menikmati keindahan ITB, Anda bertemu
Arvi yang sedang membawa sebuah mesin turing. Karena penasaran dengan mesin turing tersebut kalian mendekati Arvi. Arvi pun menjelaskan bahwa mesin turing
yang Ia punya bukanlah mesin turing biasa, melainkan mesin turing tersebut hanya dapat menerima Inputan beberapa pita yang berisikan string dengan 
character 'A' dan 'B' dan dapat mencetak sebuah string tak terhingga panjangnya yang memenuhi sebuah syarat. Syarat tersebut adalah string yang dibuat tidak boleh
mengandung string yang diberikan sebagai substringnya.

mesin tersebut biasanya dipakai oleh Arvi untuk menghilangkan kejenuhannya, sayangnya mesin tersebut sedang rusak :( . Karena merasa kasihan, Anda pun
ingin membuat program yang dapat berkerja sebagai mesin turing tersebut. Bagaimanakah program mesin turing kalian?  

\subsection*{Format Masukan}

Baris pertama berisi sebuah bilangan bulat $N$ ($1 \leq N \leq 1e5$) yang merupakan banyaknya string.
$N$ Baris berikutnya berisi string $S_i$ (dengan panjang $S_i$ : $1 \leq S_i \leq 50$). $S_i$ menyatakan string ke-$i$.

\subsection*{Format Keluaran}

Jika memiliki solusi maka tuliskan $1e5$ character pertama dari solusi. Jika tidak ada tuliskan "Tidak memiliki solusi".
\\

\begin{multicols}{2}
\subsection*{Contoh Masukan}
\begin{lstlisting}
2
A
BBB
\end{lstlisting}
\columnbreak
\subsection*{Contoh Keluaran}
\begin{lstlisting}
Tidak memiliki solusi
\end{lstlisting}
\vfill
\null
\end{multicols}

\pagebreak

\end{document}
