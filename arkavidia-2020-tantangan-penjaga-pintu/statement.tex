\documentclass{article}

\usepackage{geometry}
\usepackage{amsmath}
\usepackage{graphicx}
\usepackage{listings}
\usepackage{hyperref}
\usepackage{multicol}
\usepackage{fancyhdr}
\pagestyle{fancy}
\hypersetup{ colorlinks=true, linkcolor=black, filecolor=magenta, urlcolor=cyan}
\geometry{ a4paper, total={170mm,257mm}, top=20mm, right=20mm, bottom=20mm, left=20mm}
\setlength{\parindent}{0pt}
\setlength{\parskip}{1em}
\renewcommand{\headrulewidth}{0pt}
\lhead{Competitive Programming - Arkavidia V}
\fancyfoot[CE,CO]{\thepage}
\lstset{
    basicstyle=\ttfamily\small,
    columns=fixed,
    extendedchars=true,
    breaklines=true,
    tabsize=2,
    prebreak=\raisebox{0ex}[0ex][0ex]{\ensuremath{\hookleftarrow}},
    frame=none,
    showtabs=false,
    showspaces=false,
    showstringspaces=false,
    prebreak={},
    keywordstyle=\color[rgb]{0.627,0.126,0.941},
    commentstyle=\color[rgb]{0.133,0.545,0.133},
    stringstyle=\color[rgb]{01,0,0},
    captionpos=t,
    escapeinside={(\%}{\%)}
}

\begin{document}

\begin{center}
    \section*{Tantangan Penjaga Pintu} % ganti judul soal

    \begin{tabular}{ | c c | }
        \hline
        Batas Waktu  & 1s \\    % jangan lupa ganti time limit
        Batas Memori & 64MB \\  % jangan lupa ganti memory limit
        \hline
    \end{tabular}
\end{center}

\subsection*{Deskripsi}

Elza dan Anda , seorang Arkavidian, sedang melakukan petualangan ke hutan misterius di utara untuk menyelematkan Negara Arkavidia.
Sesampainya di depan pintu hutan misterius, kalian mendapatkan tantangan dari penjaga pintu.
kalian diberikan $N$ buah titik yang membagi rata sebuah lingkaran. selanjutnya Kalian harus berpindah dari titiknya
sekarang (titik 0) ke titik-titik lain yang memenuhi syarat:

\begin{itemize}
    \setlength\itemsep{0pt}
    \item Titik yang hendak dituju tidak bersebelahan dengan posisi sekarang
    \item Titik yang hendak dituju tidak boleh berpotongan/ coincide dengan \textit{path} sebelumnya.
\end{itemize}

Setelah itu sang penjaga pintu bertanya kepada kalian berapa banyak \textit{path} yang dapat terbentuk dari $N$ buah titik tersebut.
Saat kamu melihat muka Elza, ia terlihat kebingungan, sebagai partnernya yang setia, Anda harus membantunya 
agar dapat masuk ke hutan misterius tersebut. Dapatkah Anda menolong Elza?

\subsection*{Format Masukan}
Baris pertama terdiri dari satu bilangan bulat positif $N$ ($1 \leq N \leq 100.000$), menyatakan banyaknya point pembagi lingkaran.

\subsection*{Format Keluaran}
Sebuah baris yang menyatakan banyaknya \textit{path} yang ada.
\\

\begin{multicols}{2}
\subsection*{Contoh Masukan}
\begin{lstlisting}
5
\end{lstlisting}
\columnbreak
\subsection*{Contoh Keluaran}
\begin{lstlisting}
2
\end{lstlisting}
\vfill
\null
\end{multicols}

\subsection*{Penjelasan}
Jika setiap titik diberikan nomor 0,1,2,3,4 maka \textit{path} yang mungkin adalah 0,2,4 dan 0,3,1
% Jika dibutuhkan, tambahkan penjelasan di sini

\pagebreak

\end{document}

