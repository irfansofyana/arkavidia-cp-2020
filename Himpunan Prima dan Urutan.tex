\documentclass{article}

\usepackage{geometry}
\usepackage{amsmath}
\usepackage{graphicx}
\usepackage{listings}
\usepackage{hyperref}
\usepackage{multicol}
\usepackage{fancyhdr}
\pagestyle{fancy}
\hypersetup{ colorlinks=true, linkcolor=black, filecolor=magenta, urlcolor=cyan}
\geometry{ a4paper, total={170mm,257mm}, top=20mm, right=20mm, bottom=20mm, left=20mm}
\setlength{\parindent}{0pt}
\setlength{\parskip}{1em}
\renewcommand{\headrulewidth}{0pt}
\lhead{Competitive Programming - Arkavidia V}
\fancyfoot[CE,CO]{\thepage}
\lstset{
    basicstyle=\ttfamily\small,
    columns=fixed,
    extendedchars=true,
    breaklines=true,
    tabsize=2,
    prebreak=\raisebox{0ex}[0ex][0ex]{\ensuremath{\hookleftarrow}},
    frame=none,
    showtabs=false,
    showspaces=false,
    showstringspaces=false,
    prebreak={},
    keywordstyle=\color[rgb]{0.627,0.126,0.941},
    commentstyle=\color[rgb]{0.133,0.545,0.133},
    stringstyle=\color[rgb]{01,0,0},
    captionpos=t,
    escapeinside={(\%}{\%)}
}

\begin{document}

\begin{center}
    \section*{G. Himpunan Prima dan Urutan} % ganti judul soal

    \begin{tabular}{ | c c | }
        \hline
        Batas Waktu  & 1s \\    % jangan lupa ganti time limit
        Batas Memori & 512MB \\  % jangan lupa ganti memory limit
        \hline
    \end{tabular}
\end{center}

\subsection*{Deskripsi}

Elza sedang mencari semua bilangan yang faktor primanya merupakan upahimpunan/\textit{subset} dari sebuah himpunan $H$, dengan $H$ berisi sejumlah bilangan prima.
Anne, di sisi lain, sedang mencari urutan sebuah bilangan $K$ pada himpunan tersebut.
Tugas Anda adalah membantu Elza dan Anne untuk memecahkan masalah mereka sekaligus dengan cara \textit{programming}.

\subsection*{Format Masukan}

Baris pertama terdiri atas sebuah bilangan bulat positif $H$ ($1 \leq H \leq 100.000$).
$H$ baris berikutnya berisi sebuah bilangan prima.
Baris terakhir berisi sebuah bilangan bulat $K$

\subsection*{Format Keluaran}

Keluaran berupa posisi nilai $K$ pada deretan bilangan yang faktor primanya $\subseteq H$.
\\

\begin{multicols}{2}
\subsection*{Contoh Masukan}
\begin{lstlisting}
2
2 3
9
\end{lstlisting}
\columnbreak
\subsection*{Contoh Keluaran}
\begin{lstlisting}
6
\end{lstlisting}
\vfill
\null
\end{multicols}

% \subsection*{Penjelasan}
% Jika dibutuhkan, tambahkan penjelasan di sini

\pagebreak

\end{document}

