\documentclass{article}

\usepackage{geometry}
\usepackage{amsmath}
\usepackage{graphicx}
\usepackage{listings}
\usepackage{hyperref}
\usepackage{multicol}
\usepackage{fancyhdr}
\pagestyle{fancy}
\hypersetup{ colorlinks=true, linkcolor=black, filecolor=magenta, urlcolor=cyan}
\geometry{ a4paper, total={170mm,257mm}, top=20mm, right=20mm, bottom=20mm, left=20mm}
\setlength{\parindent}{0pt}
\setlength{\parskip}{1em}
\renewcommand{\headrulewidth}{0pt}
\lhead{Competitive Programming - Arkavidia V}
\fancyfoot[CE,CO]{\thepage}
\lstset{
    basicstyle=\ttfamily\small,
    columns=fixed,
    extendedchars=true,
    breaklines=true,
    tabsize=2,
    prebreak=\raisebox{0ex}[0ex][0ex]{\ensuremath{\hookleftarrow}},
    frame=none,
    showtabs=false,
    showspaces=false,
    showstringspaces=false,
    prebreak={},
    keywordstyle=\color[rgb]{0.627,0.126,0.941},
    commentstyle=\color[rgb]{0.133,0.545,0.133},
    stringstyle=\color[rgb]{01,0,0},
    captionpos=t,
    escapeinside={(\%}{\%)}
}

\begin{document}

\begin{center}
    \section*{XOR Beruntun} % ganti judul soal

    \begin{tabular}{ | c c | }
        \hline
        Batas Waktu  & 1s \\    % jangan lupa ganti time limit
        Batas Memori & 32MB \\  % jangan lupa ganti memory limit
        \hline
    \end{tabular}
\end{center}

\subsection*{Deskripsi}
Diberikan sebuah array $A$ dengan panjang $N$.
Terdapat sebuah fungsi $f$ yang menerima masukan sebuah array.
Fungsi $f$ akan menghitung nilai XOR beruntun dari sebuah array.
XOR beruntun adalah operasi XOR bersebelahan yang dilakukan pada array hingga panjangnya 1.
Jika terdapat array $A$ dengan elemen $A_{i}$ ($1 \leq i \leq N$), 
setelah dilakukan operasi XOR bersebelahan pada array $A$, 
maka array $A$ akan menjadi array $B$ dengan elemen $B_{i}$ ($1 \leq i \leq N-1$), 
$B_{i} = A_{i} XOR A_{i + 1}$.
Hitunglah $f(A)$.

\subsection*{Format Masukan}

Pada baris pertama diberikan $N$.
Pada baris kedua diberikan array $A$ dengan panjang $N$.

\subsection*{Format Keluaran}

Keluarkan sebuah integer $X$ yang merupakan hasil $f(A)$.
\\

\begin{multicols}{2}
\subsection*{Contoh Masukan}
\begin{lstlisting}
3
1 3 5
\end{lstlisting}
\columnbreak
\subsection*{Contoh Keluaran}
\begin{lstlisting}
4
\end{lstlisting}
\vfill
\null
\end{multicols}

\subsection*{Penjelasan}
Constraint:
$(1 \leq N \leq 2.10^5)$,
$(0 \leq Ai \leq 10^9)$,
$(1 \leq i \leq N)$.

\pagebreak.

\end{document}
