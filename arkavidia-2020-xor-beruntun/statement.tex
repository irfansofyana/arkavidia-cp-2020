\documentclass{article}

\usepackage{geometry}
\usepackage{amsmath}
\usepackage{graphicx}
\usepackage{listings}
\usepackage{hyperref}
\usepackage{multicol}
\usepackage{fancyhdr}
\pagestyle{fancy}
\hypersetup{ colorlinks=true, linkcolor=black, filecolor=magenta, urlcolor=cyan}
\geometry{ a4paper, total={170mm,257mm}, top=20mm, right=20mm, bottom=20mm, left=20mm}
\setlength{\parindent}{0pt}
\setlength{\parskip}{1em}
\renewcommand{\headrulewidth}{0pt}
\lhead{Competitive Programming - Arkavidia VI}
\fancyfoot[CE,CO]{\thepage}
\lstset{
    basicstyle=\ttfamily\small,
    columns=fixed,
    extendedchars=true,
    breaklines=true,
    tabsize=2,
    prebreak=\raisebox{0ex}[0ex][0ex]{\ensuremath{\hookleftarrow}},
    frame=none,
    showtabs=false,
    showspaces=false,
    showstringspaces=false,
    prebreak={},
    keywordstyle=\color[rgb]{0.627,0.126,0.941},
    commentstyle=\color[rgb]{0.133,0.545,0.133},
    stringstyle=\color[rgb]{01,0,0},
    captionpos=t,
    escapeinside={(\%}{\%)}
}

\begin{document}

\begin{center}
    \section*{XOR Beruntun} % ganti judul soal

    \begin{tabular}{ | c c | }
        \hline
        Batas Waktu  & 1s \\    % jangan lupa ganti time limit
        Batas Memori & 64MB \\  % jangan lupa ganti memory limit
        \hline
    \end{tabular}
\end{center}

\subsection*{Deskripsi}

Suatu saat pada sore hari, Elza bersama $N$ orang temannya melaksanakan arisan bulanan.
Karena bosan dengan pengundian menggunakan kertas, Elza bersama temannya membuat sebuah \textit{pseudo random generator}.
\textit{pseudo random generator} bekerja dengan cara memasukkan banyaknya peserta arisan, yaitu $N$, lalu peserta arisan bergantian memasukkan sebuah bilangan bulat positif $X$.
Selanjutnya, \textit{pseudo random generator} akan melakukan operasi XOR bersebelahan.

Operasi XOR bersebelahan yang dimaksud adalah sebagai berikut:

Bilangan yang dimasukkan peserta arisan pertama akan dilakukan proses XOR dengan bilangan peserta arisan kedua, 
Bilangan yang dimasukkan peserta arisan kedua akan dilakukan proses XOR dengan peserta arisan ketiga, 
dan begitu seterusnya hingga bilangan peserta arisan ke-$N - 1$ dilakukan proses XOR dengan bilangan peserta ke-$N$. Proses ini akan menghasilkan sebuah barisan bilangan baru berisi $N - 1$ bilangan yang merupakan hasil operasi XOR bersebelahan. 
Pemrosesan tersebut dilakukan terus menerus hingga didapatkan barisan bilangan yang terdiri satu bilangan saja.

Anda yang sangat jago membuat program diminta Elza untuk membuatkan program pseudo random generator dan menampilkan bilangan terakhir yang dihasilkan.

\subsection*{Format Masukan}

Baris pertama berisi sebuah bilangan bulat $N$ ($1 \leq N \leq 200.000$) yang merupakan banyaknya teman Elza.
Baris kedua berisi $N$ buah bilangan bulat $X_i$ ($1 \leq X_i \leq 10^{9}$) dipisahkan dengan spasi. $X_i$ menyatakan bilangan yang dimasukkan teman Elza ke-$i$.

\subsection*{Format Keluaran}

Sebuah bilangan bulat $Y$ hasil dari  \textit{pseudo random generator}.
\\

\begin{multicols}{2}
\subsection*{Contoh Masukan}
\begin{lstlisting}
3
1 3 5
\end{lstlisting}
\columnbreak
\subsection*{Contoh Keluaran}
\begin{lstlisting}
4
\end{lstlisting}
\vfill
\null
\end{multicols}

\subsection*{Penjelasan}
Bilangan yang dimasukkan peserta arisan adalah $1, 3, 5$. 
Setelah satu kali operasi XOR bersebelahan, maka barisan yang dihasilkan adalah $[(1$ XOR $3), (3$ XOR $5)]$ atau  $[2, 6]$. Setelah melakukan operasi XOR bersebelahan sekali lagi, maka barisan bilangan yang dihasilkan adalah $[2$ XOR $6]$ atau [4]. Sehingga bilangan yang dihasilkan \textit{pseudo random generator} adalah 4.


\pagebreak

\end{document}
