\documentclass{article}

\usepackage{geometry}
\usepackage{amsmath}
\usepackage{graphicx}
\usepackage{listings}
\usepackage{hyperref}
\usepackage{multicol}
\usepackage{fancyhdr}
\pagestyle{fancy}
\hypersetup{ colorlinks=true, linkcolor=black, filecolor=magenta, urlcolor=cyan}
\geometry{ a4paper, total={170mm,257mm}, top=20mm, right=20mm, bottom=20mm, left=20mm}
\setlength{\parindent}{0pt}
\setlength{\parskip}{1em}
\renewcommand{\headrulewidth}{0pt}
\lhead{Competitive Programming - Arkavidia V}
\fancyfoot[CE,CO]{\thepage}
\lstset{
    basicstyle=\ttfamily\small,
    columns=fixed,
    extendedchars=true,
    breaklines=true,
    tabsize=2,
    prebreak=\raisebox{0ex}[0ex][0ex]{\ensuremath{\hookleftarrow}},
    frame=none,
    showtabs=false,
    showspaces=false,
    showstringspaces=false,
    prebreak={},
    keywordstyle=\color[rgb]{0.627,0.126,0.941},
    commentstyle=\color[rgb]{0.133,0.545,0.133},
    stringstyle=\color[rgb]{01,0,0},
    captionpos=t,
    escapeinside={(\%}{\%)}
}

\begin{document}

\begin{center}
    \section*{XOR Beruntun} % ganti judul soal

    \begin{tabular}{ | c c | }
        \hline
        Batas Waktu  & 1s \\    % jangan lupa ganti time limit
        Batas Memori & 32MB \\  % jangan lupa ganti memory limit
        \hline
    \end{tabular}
\end{center}

\subsection*{Deskripsi}

Suatu saat pada sore hari, Elza bersama $N$ temannya melaksanakan arisan bulanan.
Karena bosan dengan pengundian menggunakan kertas, Elza bersama temannya membuat sebuah pseudo random generator.
Pseudo random generator bekerja dengan cara memasukkan banyaknya peserta arisan, $N$, lalu peserta arisan bergantian memasukkan sebuah bilangan $X$.
Selanjutnya, pseudo random generator akan melakukan operasi $XOR$ bersebelahan.

Bilangan yang dimasukkan peserta arisan pertama akan dilakukan proses $XOR$ dengan bilangan peserta arisan kedua, 
Bilangan yang dimasukkan peserta arisan kedua akan dilakukan proses $XOR$ dengan peserta arisan ketiga, 
dan begitu seterusnya hingga bilangan peserta arisan ke-$N - 1$ dilakukan proses $XOR$ dengan bilangan peserta ke-$N$.

Proses ini akan menghasilkan sebuah barisan bilangan baru berisi $N - 1$ bilangan yang merupakan hasil operasi $XOR$ bersebelahan. 
Pemrosesan tersebut dilakukan terus menerus hingga didapatkan satu bilangan saja.

Anda yang sangat jago membuat program diminta Elza untuk membuatkan program pseudo random generator dan menampilkan bilangan terakhir yang dihasilkan.

\subsection*{Format Masukan}

Baris pertama berisi sebuah bilangan bulat $N$ ($1 \leq N \leq 2.10^5$) yang merupakan banyaknya teman Elza.
Baris kedua berisi $N$ buah bilangan bulat $X$ ($1 \leq N \leq 10^9$) dipisahkan dengan spasi. $X_i$ menyatakan bilangan yang dimasukkan teman Elza ke-$i$.

\subsection*{Format Keluaran}

Sebuah bilangan bulat $Y$ hasil dari pseudo random generator.
\\

\begin{multicols}{2}
\subsection*{Contoh Masukan}
\begin{lstlisting}
3
1 3 5
\end{lstlisting}
\columnbreak
\subsection*{Contoh Keluaran}
\begin{lstlisting}
4
\end{lstlisting}
\vfill
\null
\end{multicols}

% \subsection*{Penjelasan}
% Jika dibutuhkan, tambahkan penjelasan di sini

\pagebreak

\end{document}

