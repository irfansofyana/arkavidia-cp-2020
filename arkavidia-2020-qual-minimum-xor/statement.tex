\documentclass{article}

\usepackage{geometry}
\usepackage{amsmath}
\usepackage{graphicx}
\usepackage{listings}
\usepackage{hyperref}
\usepackage{multicol}
\usepackage{fancyhdr}
\pagestyle{fancy}
\hypersetup{ colorlinks=true, linkcolor=black, filecolor=magenta, urlcolor=cyan}
\geometry{ a4paper, total={170mm,257mm}, top=20mm, right=20mm, bottom=20mm, left=20mm}
\setlength{\parindent}{0pt}
\setlength{\parskip}{1em}
\renewcommand{\headrulewidth}{0pt}
\lhead{Competitive Programming - Arkavidia VI}
\fancyfoot[CE,CO]{\thepage}
\lstset{
    basicstyle=\ttfamily\small,
    columns=fixed,
    extendedchars=true,
    breaklines=true,
    tabsize=2,
    prebreak=\raisebox{0ex}[0ex][0ex]{\ensuremath{\hookleftarrow}},
    frame=none,
    showtabs=false,
    showspaces=false,
    showstringspaces=false,
    prebreak={},
    keywordstyle=\color[rgb]{0.627,0.126,0.941},
    commentstyle=\color[rgb]{0.133,0.545,0.133},
    stringstyle=\color[rgb]{01,0,0},
    captionpos=t,
    escapeinside={(\%}{\%)}
}

\begin{document}

\begin{center}
    \section*{Minimum XOR} % ganti judul soal

    \begin{tabular}{ | c c | }
        \hline
        Batas Waktu  & 1s \\    % jangan lupa ganti time limit
        Batas Memori & 64MB \\  % jangan lupa ganti memory limit
        \hline
    \end{tabular}
\end{center}

\subsection*{Deskripsi}
XOR atau $\oplus$ merupakan salah satu operator logika dalam aljabar boolean yang menerima dua buah operan berupa nilai boolean ($true, false$ atau $0, 1$) dan menghasilkan nilai $true$ atau $1$ jika (dan hanya jika) salah satu operan bernilai $true$ atau $1$ tetapi tidak keduanya. Tabel kebenaran berikut menunjukkan nilai $A \oplus B$ untuk nilai $A$ dan $B$ yang berbeda-beda: 

\begin{center}
\begin{tabular}{c|c|c}
$A$ & $B$ & $A \oplus  B$ \\ \hline
0 & 0 & 0       \\
1 & 0 & 1       \\
0 & 1 & 1       \\
1 & 1 & 0
\end{tabular}
\end{center}
Operasi XOR juga dapat dilakukan dengan operan berupa bilangan bulat non-negatif, dengan cara melakukan operasi XOR pada tiap digit kedua operan dalam bentuk biner dengan posisi yang bersesuaian. Contohnya:
$$ 181 \oplus 108 = 10110101_2 \oplus 01101100_2 = 11011001_2 = 217 $$

Diberikan sebuah bilangan bulat non-negatif $N$ dan sebuah bilangan asli $k$. Untuk sembarang tupel $k$ bilangan bulat non-negatif $(a_1, a_2, \ldots, a_k)$ yang berjumlah $N$, carilah nilai minimum dari $a_1 \oplus a_2 \oplus \ldots \oplus a_k$ (perhatikan bahwa operasi XOR bersifat asosiatif sehingga urutan pengoperasian pada persamaan tersebut tidak berpengaruh pada hasil akhir).

\subsection*{Format Masukan}

Masukan terdiri dari dua bilangan  $N, k$ ($0 \leq N \leq 10^{18}, 1 \leq k \leq 10^{18}$)

\subsection*{Format Keluaran}

Keluaran berupa nilai minimum yang mungkin dihasilkan oleh $a_1 \oplus a_2 \oplus \ldots \oplus a_k$ dimana $a_i$ merupakan bilangan bulat non-negatif yang berjumlah $N$.

\begin{multicols}{2}
\subsection*{Contoh Masukan}
\begin{lstlisting}
5 3
\end{lstlisting}
\columnbreak
\subsection*{Contoh Keluaran}
\begin{lstlisting}
1
\end{lstlisting}
\vfill
\null
\end{multicols}

\subsection*{Penjelasan}
Bilangan $2, 2, 1$ berjumlah $5$ dengan nilai $2 \oplus 2 \oplus 1 = 1$. Dapat dibuktikan tidak dapat dihasilkan nilai XOR yang lebih kecil dari $1$.
\pagebreak

\end{document}
