\documentclass{article}

\usepackage{geometry}
\usepackage{amsmath}
\usepackage{graphicx}
\usepackage{listings}
\usepackage{hyperref}
\usepackage{multicol}
\usepackage{fancyhdr}
\pagestyle{fancy}
\hypersetup{ colorlinks=true, linkcolor=black, filecolor=magenta, urlcolor=cyan}
\geometry{ a4paper, total={170mm,257mm}, top=20mm, right=20mm, bottom=20mm, left=20mm}
\setlength{\parindent}{0pt}
\setlength{\parskip}{1em}
\renewcommand{\headrulewidth}{0pt}
\lhead{Competitive Programming - Arkavidia VI}
\fancyfoot[CE,CO]{\thepage}

\begin{document}

\begin{center}
    \section*{Kelereng Arvy dan Elza}

    \begin{tabular}{ | c c | }
        \hline
        Batas Waktu  & 1s \\
        Batas Memori & 64MB \\
        \hline
    \end{tabular}
\end{center}

\subsection*{Deskripsi}

Ada dua anak terkenal di Negeri Arkavy yang senang bermain kelereng. Mereka adalah
Arvy dan Elza. Arvy memiliki kelereng sejumlah $x$ dan Elza memiliki kelereng sejumlah $y$.
Arvy ingin tahu berapa jumlah kelereng yang dimiliki oleh dirinya dan juga Elza. Sebagai teman yang baik, dapatkah Anda 
membantu Arvy?

\subsection*{Format Masukan}

Sebuah baris berisi bilangan bulat $x$ dan $y$ ($1 \leq x, y \leq 100$), yaitu jumlah kelereng yang dimiliki masing-masing 
oleh Arvy dan Elza.

\subsection*{Format Keluaran}

Keluarkan sebuah baris berisi bilangan bulat yang menyatakan jumlah kelereng yang dimiliki oleh Arvy dan Elza
\\

\begin{multicols}{2}
\subsection*{Contoh Masukan}
\begin{lstlisting}
1 2
\end{lstlisting}
\columnbreak
\subsection*{Contoh Keluaran}
\begin{lstlisting}
3
\end{lstlisting}
\vfill
\null
\end{multicols}

% \subsection*{Penjelasan}
% Jika dibutuhkan, tambahkan penjelasan di sini

\pagebreak

\end{document}