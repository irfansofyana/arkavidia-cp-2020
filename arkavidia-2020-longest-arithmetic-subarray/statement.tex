\documentclass{article}

\usepackage{geometry}
\usepackage{amsmath}
\usepackage{graphicx}
\usepackage{listings}
\usepackage{hyperref}
\usepackage{multicol}
\usepackage{fancyhdr}
\pagestyle{fancy}
\hypersetup{ colorlinks=true, linkcolor=black, filecolor=magenta, urlcolor=cyan}
\geometry{ a4paper, total={170mm,257mm}, top=20mm, right=20mm, bottom=20mm, left=20mm}
\setlength{\parindent}{0pt}
\setlength{\parskip}{1em}
\renewcommand{\headrulewidth}{0pt}
\lhead{Competitive Programming - Arkavidia VI}
\fancyfoot[CE,CO]{\thepage}
\lstset{
    basicstyle=\ttfamily\small,
    columns=fixed,
    extendedchars=true,
    breaklines=true,
    tabsize=2,
    prebreak=\raisebox{0ex}[0ex][0ex]{\ensuremath{\hookleftarrow}},
    frame=none,
    showtabs=false,
    showspaces=false,
    showstringspaces=false,
    prebreak={},
    keywordstyle=\color[rgb]{0.627,0.126,0.941},
    commentstyle=\color[rgb]{0.133,0.545,0.133},
    stringstyle=\color[rgb]{01,0,0},
    captionpos=t,
    escapeinside={(\%}{\%)}
}

\begin{document}

\begin{center}
    \section*{Snowman} % ganti judul soal

    \begin{tabular}{ | c c | }
        \hline
        Batas Waktu  & 1s \\    % jangan lupa ganti time limit
        Batas Memori & 64MB \\  % jangan lupa ganti memory limit
        \hline
    \end{tabular}
\end{center}

\subsection*{Deskripsi}

Elza dan Anne adalah 2 orang anak kecil dari sebuah kerajaan yang sangat jauh.
Pada suatu malam, saat Elza dan Anne sedang membuat \textit{snowman}, Elza mendengarkan sebuah bisikan misterius.
bisikan itu memberikan sebuah ide permainan baru. Elza awalnya membuat $N$ buah \textit{snowman}.
Setelah itu, dia memberi bilangan $A_i$ pada setiap \textit{snowman} ke-$i$.
Selanjutnya Elza memberikan 2 buah bilangan $l$ dan $r$. Ia meminta Anne untuk menghitung berapa panjang maksimal 
\textit{snowman} yang berurutan di dalam rentang $l$ dan $r$ (inklusif) dengan selisih bilangan pada \textit{snowman} yang 
berurutan adalah konstan. Tetapi karena Anne masih belum mahir menghitung, 
ia meminta Anda sebagai pengawal pribadi dari mereka untuk menyelesaikannya.
Bantulah Anne untuk menyelesaikan permainan dari Elza!

\subsection*{Format Masukan}

Baris pertama terdiri dari satu bilangan bulat positif $N$ ($1 \leq N \leq 100.000$), menyatakan banyaknya \textit{snowman}.
Baris kedua terdiri dari N bilangan bulat $A_i$ ($-10^{9} \leq A_i \leq 10^{9}$), menyatakan bilangan pada \textit{snowman} ke-$i$.
Baris ketiga terdiri dari sebuah bilangan bulat positif $Q$ ($1 \leq Q \leq 100.000$), menyatakan banyaknya pertanyaan Elza.
$Q$ baris berikutnya terdiri dari 2 bilangan, dengan baris ke-$i$ menyatakan bilangan $l_i$ dan $r_i$ ($1 \leq l_i \leq r_i \leq N$).

\subsection*{Format Keluaran}
Untuk setiap pertanyaan Elza, keluarkan panjang maksimal \textit{snowman} yang berurutan di dalam rentang $l_i$ dan $r_i$ (inklusif) 
dengan selisih bilangan pada \textit{snowman} yang berurutan adalah konstan.
\\

\begin{multicols}{2}
\subsection*{Contoh Masukan}
\begin{lstlisting}
8
1 3 5 6 7 8 10 12
3
1 8
2 4
6 6
\end{lstlisting}
\columnbreak
\subsection*{Contoh Keluaran}
\begin{lstlisting}
4
2
1
\end{lstlisting}
\vfill
\null
\end{multicols}

\subsection*{Penjelasan}
1. [5,6,7,8] $-->$ jaraknya 1, panjangnya 4

2. [3,5] atau [5,6] $-->$ jaraknya 2, panjangnya 2

3. [8] $-->$ panjangnya 1

% Jika dibutuhkan, tambahkan penjelasan di sini

\pagebreak

\end{document}

