\documentclass{article}

\usepackage{geometry}
\usepackage{amsmath}
\usepackage{graphicx}
\usepackage{listings}
\usepackage{hyperref}
\usepackage{multicol}
\usepackage{fancyhdr}
\pagestyle{fancy}
\hypersetup{ colorlinks=true, linkcolor=black, filecolor=magenta, urlcolor=cyan}
\geometry{ a4paper, total={170mm,257mm}, top=20mm, right=20mm, bottom=20mm, left=20mm}
\setlength{\parindent}{0pt}
\setlength{\parskip}{1em}
\renewcommand{\headrulewidth}{0pt}
\lhead{Competitive Programming - Arkavidia VI}
\fancyfoot[CE,CO]{\thepage}
\lstset{
    basicstyle=\ttfamily\small,
    columns=fixed,
    extendedchars=true,
    breaklines=true,
    tabsize=2,
    prebreak=\raisebox{0ex}[0ex][0ex]{\ensuremath{\hookleftarrow}},
    frame=none,
    showtabs=false,
    showspaces=false,
    showstringspaces=false,
    prebreak={},
    keywordstyle=\color[rgb]{0.627,0.126,0.941},
    commentstyle=\color[rgb]{0.133,0.545,0.133},
    stringstyle=\color[rgb]{01,0,0},
    captionpos=t,
    escapeinside={(\%}{\%)}
}

\begin{document}

\begin{center}
    \section*{Virtual Contest} % ganti judul soal

    \begin{tabular}{ | c c | }
        \hline
        Batas Waktu  & 1s \\    % jangan lupa ganti time limit
        Batas Memori & 64MB \\  % jangan lupa ganti memory limit
        \hline
    \end{tabular}
\end{center}

\subsection*{Deskripsi}
Di Negeri Arkavy sedang diadakan sebuah kontes pemrograman antar penduduk. Elza yang merupakan penggemar 
pemrograman pun ingin mengikuti kontes pemrograman paling bergengsi di Negerinya ini. Sayangnya, saat waktu kontes tiba,
Elza mendadak sakit hingga tidak dapat mengikut kontes pemrograman tersebut. Anne selaku ketua penyelenggara kontes 
pemrograman di Negeri Arkavy memang baik hati. Dia menyediakan sebuah kontes virtual, dimana seseorang dapat mengikuti 
kontes pemrograman tersebut setelah kontes pemrograman non-virtualnya telah selesai dilaksanakan. Elza pun cukup bahagia mendengar
kabar tersebut dan dia memutuskan akan mengikuti kontes virtual tersebut.

Peringkat pada kontes pemrograman (baik yang virtual maupun tidak) di Negeri Arkavy ditentukan oleh skor, skor yang paling tinggi akan mendapat peringkat satu. 
Peringkat peserta pada kontes pemrograman non-virtual akan dimasukkan ke peringkat kontes pemrograman virtual. Menyadari sistem 
peringkat kontes virtual seperti ini, Elza lantas berpikir peringkatnya mungkin bisa saja berubah-ubah seiringnya ada peserta baru 
yang mengikuti kontes virtual. Dirinya pun sekarang penasaran, jika ada beberapa orang yang mengikuti kontes virtual tersebut dan mendapatkan skor tertentu, dia 
sesekali akan bertanya berapa skor yang dimiliki oleh peringkat $X$. Sebagai teman baik Elza, Anda tentu ingin membantunya.

\subsection*{Format Masukan}
Baris pertama berisi sebuah bilangan bulat $N$ ($1 \leq N \leq 20.000$) yang menyatakan banyaknya peserta kontes pemrograman non-virtual.
Baris kedua terdapat $N$ buah angka $a_i$ ($1 \leq a_i \leq 10^9$), yang menyatakan skor peserta kontes pemrograman non-virtual \textbf{secara terurut}
Baris ketiga berisi bilangan bulat $Q$ ($1 \leq Q \leq 30.000$), yakni banyaknya \textit{query} yang sedang dipikirkan Elza
$Q$ Baris berikutnya berisi salah satu dari 2 tipe input :
\begin{itemize}
    \setlength{\itemsep}{0pt}
    \item Input berupa "1 $X$" yang berarti terdapat peserta virtual contest baru dan mendapat skor $X$.
    \item Input berupa "2 $X$" yang berarti Elza menanyakan skor orang dengan peringkat $X$ saat itu.
\end{itemize}
Untuk \textit{query} tipe-2 dapat dipastikan bahwa banyak peserta yang masuk dalam peringkat kontes virtual lebih
dari sama dengan peringkat yang ditanya.

\subsection*{Format Keluaran}

Untuk setiap query berupa "2 $X$", tuliskan nilai peserta pada peringkat tersebut.
\\

\begin{multicols}{2}
\subsection*{Contoh Masukan}
\begin{lstlisting}
6
98 66 59 57 34 12
6
2 3
1 61
2 3
1 98
2 1
2 2
\end{lstlisting}
\columnbreak
\subsection*{Contoh Keluaran}
\begin{lstlisting}
59
61
98
98
\end{lstlisting}
\vfill
\null
\end{multicols}

\subsection*{Penjelasan}
Asumsikan $x$ tidak melebihi jumlah peserta saat itu

\pagebreak

\end{document}

