\documentclass{article}

\usepackage{geometry}
\usepackage{amsmath}
\usepackage{graphicx}
\usepackage{listings}
\usepackage{hyperref}
\usepackage{multicol}
\usepackage{fancyhdr}
\pagestyle{fancy}
\hypersetup{ colorlinks=true, linkcolor=black, filecolor=magenta, urlcolor=cyan}
\geometry{ a4paper, total={170mm,257mm}, top=20mm, right=20mm, bottom=20mm, left=20mm}
\setlength{\parindent}{0pt}
\setlength{\parskip}{1em}
\renewcommand{\headrulewidth}{0pt}
\lhead{Competitive Programming - Arkavidia VI}
\fancyfoot[CE,CO]{\thepage}
\lstset{
    basicstyle=\ttfamily\small,
    columns=fixed,
    extendedchars=true,
    breaklines=true,
    tabsize=2,
    prebreak=\raisebox{0ex}[0ex][0ex]{\ensuremath{\hookleftarrow}},
    frame=none,
    showtabs=false,
    showspaces=false,
    showstringspaces=false,
    prebreak={},
    keywordstyle=\color[rgb]{0.627,0.126,0.941},
    commentstyle=\color[rgb]{0.133,0.545,0.133},
    stringstyle=\color[rgb]{01,0,0},
    captionpos=t,
    escapeinside={(\%}{\%)}
}

\begin{document}

\begin{center}
    \section*{Virtual Contest} % ganti judul soal

    \begin{tabular}{ | c c | }
        \hline
        Batas Waktu  & 1s \\    % jangan lupa ganti time limit
        Batas Memori & 64MB \\  % jangan lupa ganti memory limit
        \hline
    \end{tabular}
\end{center}

\subsection*{Deskripsi}

Elza dan Anne sedang mengikuti lomba \textit{Competitive Programming} Arkavidia. 
Anne datang lomba terlambat. Tidak lama kemudian, Elza menyelesaikan lombanya.
Elza kemudian ingin melihat peringkat dia dan Anne pada lomba tersebut.
Tugas anda adalah membuat program yang menginput kedalam \textit{scoreboard} serta menampilkan nilai pada peringkat tertentu.

\subsection*{Format Masukan}

Baris pertama merupakan bilangan bulat $N$ ($1 \leq N \leq 20.000$) yang menyatakan banyaknya peserta awal.
Baris kedua terdapat $N$ buah angka $a_n$ ($1 \leq a_n \leq 10^9$), yang menyatakan skor peserta \textbf{secara terurut}
Baris ketiga merupakan bilangan bulat $Q$ ($1 \leq Q \leq 30.000$), yakni banyaknya \textit{query} yang dimasukkan pada \textit{scoreboard}
$Q$ Baris berikutnya berisi salah satu dari 2 tipe input :
\begin{itemize}
    \setlength{\itemsep}{0pt}
    \item Input berupa "1 $x$" yang berarti terdapat peserta virtual contest baru dengan skor $x$.
    \item Input berupa "2 $x$" yang berarti menanyakan skor orang dengan peringkat x saat itu.
\end{itemize}

\subsection*{Format Keluaran}

Untuk setiap query berupa "2 x", tuliskan nilai peserta pada peringkat tersebut.
\\

\begin{multicols}{2}
\subsection*{Contoh Masukan}
\begin{lstlisting}
6
98 66 59 57 34 12
6
2 3
1 61
2 3
1 98
2 1
2 2
\end{lstlisting}
\columnbreak
\subsection*{Contoh Keluaran}
\begin{lstlisting}
59
61
98
98
\end{lstlisting}
\vfill
\null
\end{multicols}

\subsection*{Penjelasan}
Asumsikan $x$ tidak melebihi jumlah peserta saat itu

\pagebreak

\end{document}

