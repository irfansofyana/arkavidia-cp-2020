\documentclass{article}

\usepackage{geometry}
\usepackage{amsmath}
\usepackage{graphicx}
\usepackage{listings}
\usepackage{hyperref}
\usepackage{multicol}
\usepackage{fancyhdr}
\pagestyle{fancy}
\hypersetup{ colorlinks=true, linkcolor=black, filecolor=magenta, urlcolor=cyan}
\geometry{ a4paper, total={170mm,257mm}, top=20mm, right=20mm, bottom=20mm, left=20mm}
\setlength{\parindent}{0pt}
\setlength{\parskip}{1em}
\renewcommand{\headrulewidth}{0pt}
\lhead{Competitive Programming - Arkavidia V}
\fancyfoot[CE,CO]{\thepage}
\lstset{
    basicstyle=\ttfamily\small,
    columns=fixed,
    extendedchars=true,
    breaklines=true,
    tabsize=2,
    prebreak=\raisebox{0ex}[0ex][0ex]{\ensuremath{\hookleftarrow}},
    frame=none,
    showtabs=false,
    showspaces=false,
    showstringspaces=false,
    prebreak={},
    keywordstyle=\color[rgb]{0.627,0.126,0.941},
    commentstyle=\color[rgb]{0.133,0.545,0.133},
    stringstyle=\color[rgb]{01,0,0},
    captionpos=t,
    escapeinside={(\%}{\%)}
}

\begin{document}

\begin{center}
    \section*{H. Triple Phytagoras} % ganti judul soal

    \begin{tabular}{ | c c | }
        \hline
        Batas Waktu  & 2s \\    % jangan lupa ganti time limit
        Batas Memori & 512MB \\  % jangan lupa ganti memory limit
        \hline
    \end{tabular}
\end{center}

\subsection*{Deskripsi}

Pak Joni sangat menyukai bilangan. Beberapa dari bilangan yang ia sukai adalah bilangan-bilangan yang membentuk Triple Pythagoras. Triple Phytagoras adalah sebuah triple ($a, b, c$) dengan $a, b, c$ adalah bilangan asli sedemikian rupa sehingga $a^2 + b^2 = c^2$. Tetapi, ia tidak menyukai Triple Phytagoras yang palsu, yaitu Triple Phytagoras yang semua bilangan pembentuknya tidak saling relatif prima dan dapat dibagi bilangan bulat lebih dari 1 tertentu sehingga mendapatkan Triple Phytagoras lain. Sebagai contoh ($6, 8, 10$) adalah Triple Phytagoras palsu karena bilangan pembentuknya tidak saling relatif prima dan jika tiap bilangan pembentuknya dibagi 2 maka akan menghasilkan ($3, 4, 5$) yang juga merupakan Triple Phytagoras.

Pak Joni memiliki bilangan $N$ yang ganjil. Bantulah Pak Joni untuk mencari banyaknya Triple Phytagoras tidak palsu yang memiliki bentuk ($N, A, B$)!


\subsection*{Format Masukan}

Masukan berupa satu bilangan $N$.

\subsection*{Format Keluaran}

Keluaran berupa satu bilangan yang menyatakan banyaknya Triple Phytagoras ($N, A, B$) yang tidak palsu.
\\

\begin{multicols}{2}
\subsection*{Contoh Masukan}
\begin{lstlisting}
3

\end{lstlisting}
\columnbreak
\subsection*{Contoh Keluaran}
\begin{lstlisting}
1
\end{lstlisting}
\vfill
\null
\end{multicols}

\subsection*{Penjelasan}
Terdapat satu Triple Phytagoras tidak palsu yang berbentuk ($3, A, B$), yaitu ($3, 4, 5$).

\pagebreak

\end{document}

