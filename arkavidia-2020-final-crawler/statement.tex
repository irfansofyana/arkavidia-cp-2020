\documentclass{article}

\usepackage{geometry}
\usepackage{amsmath}
\usepackage{graphicx}
\usepackage{listings}
\usepackage{hyperref}
\usepackage{multicol}
\usepackage{fancyhdr}
\pagestyle{fancy}
\hypersetup{ colorlinks=true, linkcolor=black, filecolor=magenta, urlcolor=cyan}
\geometry{ a4paper, total={170mm,257mm}, top=20mm, right=20mm, bottom=20mm, left=20mm}
\setlength{\parindent}{0pt}
\setlength{\parskip}{1em}
\renewcommand{\headrulewidth}{0pt}
\lhead{Competitive Programming - Arkavidia VI}
\fancyfoot[CE,CO]{\thepage}
\lstset{
    basicstyle=\ttfamily\small,
    columns=fixed,
    extendedchars=true,
    breaklines=true,
    tabsize=2,
    prebreak=\raisebox{0ex}[0ex][0ex]{\ensuremath{\hookleftarrow}},
    frame=none,
    showtabs=false,
    showspaces=false,
    showstringspaces=false,
    prebreak={},
    keywordstyle=\color[rgb]{0.627,0.126,0.941},
    commentstyle=\color[rgb]{0.133,0.545,0.133},
    stringstyle=\color[rgb]{01,0,0},
    captionpos=t,
    escapeinside={(\%}{\%)}
}

\begin{document}

\begin{center}
    \section*{Keliling ITB} % ganti judul soal

    \begin{tabular}{ | c c | }
        \hline
        Batas Waktu  & 1s \\    % jangan lupa ganti time limit
        Batas Memori & 64MB \\  % jangan lupa ganti memory limit
        \hline
    \end{tabular}
\end{center}

\subsection*{Deskripsi}
Pada hari ini Anda dan kelompok Anda datang ke ITB untuk mengikuti Final \textit{Competitive Programming} dari Arkavidia. Karena Kalian (mungkin) belum pernah
datang ke ITB, Kalian pun berkeliling ITB. Di ITB terdapat banyak sekali gedung seperti Labtek kembar (salah satu labnya adalah tempat Kalian berada sekarang).
Saat kalian tiba di Intel (Indonesia Tenggelam), di tengah Labtek Kembar, kalian bertemu dengan salah satu panitia Arkavidia yang sedang membagikan makanan gratis (semoga ada, amin).
Syarat dari makanan tersebut adalah menjawab kuis dari panitia tersebut.

kalian pun ingin mendapatkan makanan tersebut sehingga kalian pun mengikuti kuis tersebut. Setelah kalian mendaftar kalian diberikan sebuah peta ITB yang setiap gedungnya
digambarkan dengan lingkaran dengan yang bertuliskan node ke-$i$, dengan $0 \leq i$ yang terlihat seperti graph berarah. Kuisnya adalah menghitung banyaknya kemungkinan suatu gedung 
dikunjungi dalam $k$ kali langkah. Misalkan Peta Anda memiliki 3 node (node 0, node 1, dan node 2) yang setiap notenya saling terhubung (Setiap node terhubung dengan node lain), 
maka kemungkinan kombinasi bergerak dengan $k = 2 $ yaitu:

\begin{enumerate}
    \setlength\itemsep{0pt}
    \item node 0 -> node 1 -> node 0
    \item node 0 -> node 1 -> node 2
    \item node 0 -> node 2 -> node 0
    \item node 0 -> node 2 -> node 1
\end{enumerate}

sehingga node 0 dapat dikunjungi 6x, node 1 dapat dikunjungi 3x, dan node 2 dapat dikunjungi 3x. Anda pun langsung mengerti dengan soal dan langsung 
mendapatkan jawabannya. Anda pun langsung mengeluarkan laptop kesayangan Anda dan mulai mengoding. Dapatkah Anda mendapatkan makanan gratis tersebut?

\subsection*{Format Masukan}

Baris pertama berisi sebuah bilangan bulat $N$, $M$, dan $K$, ($1 \leq N,M,K \leq 100.000$). Dengan $N$ merupakan banyaknya Node, $M$ banyaknya \textit{path} yang ada.
dan $K$ banyaknya langkah.
$M$ baris berikutnya berisikan bilangan bulat $X_i$ dan $Y_i$ ($0 \leq X,Y \leq N$) yang dipisahkan dengan sebuah spasi.

\subsection*{Format Keluaran}

$N$ buah baris yang menunjukkan banyaknya kemungkinan suatu node dikunjungi.
\\

\begin{multicols}{2}
\subsection*{Contoh Masukan}
\begin{lstlisting}
3 6 2
0 1
0 2
1 0
1 2
2 0
2 1
\end{lstlisting}
\columnbreak
\subsection*{Contoh Keluaran}
\begin{lstlisting}
6
3
3
\end{lstlisting}
\vfill
\null
\end{multicols}

\subsection*{Penjelasan}
Bilangan yang dimasukkan peserta arisan adalah $1, 3, 5$. 
Setelah satu kali operasi XOR bersebelahan, maka barisan yang dihasilkan adalah $[(1$ XOR $3), (3$ XOR $5)]$ atau  $[2, 6]$. Setelah melakukan operasi XOR bersebelahan sekali lagi, maka barisan bilangan yang dihasilkan adalah $[2$ XOR $6]$ atau [4]. Sehingga bilangan yang dihasilkan \textit{pseudo random generator} adalah 4.


\pagebreak

\end{document}
