\documentclass{article}

\usepackage{geometry}
\usepackage{amsmath}
\usepackage{graphicx}
\usepackage{listings}
\usepackage{hyperref}
\usepackage{multicol}
\usepackage{fancyhdr}
\pagestyle{fancy}
\hypersetup{ colorlinks=true, linkcolor=black, filecolor=magenta, urlcolor=cyan}
\geometry{ a4paper, total={170mm,257mm}, top=20mm, right=20mm, bottom=20mm, left=20mm}
\setlength{\parindent}{0pt}
\setlength{\parskip}{1em}
\renewcommand{\headrulewidth}{0pt}
\lhead{Competitive Programming - Arkavidia VI}
\fancyfoot[CE,CO]{\thepage}
\lstset{
    basicstyle=\ttfamily\small,
    columns=fixed,
    extendedchars=true,
    breaklines=true,
    tabsize=2,
    prebreak=\raisebox{0ex}[0ex][0ex]{\ensuremath{\hookleftarrow}},
    frame=none,
    showtabs=false,
    showspaces=false,
    showstringspaces=false,
    prebreak={},
    keywordstyle=\color[rgb]{0.627,0.126,0.941},
    commentstyle=\color[rgb]{0.133,0.545,0.133},
    stringstyle=\color[rgb]{01,0,0},
    captionpos=t,
    escapeinside={(\%}{\%)}
}

\begin{document}

\begin{center}
    \section*{Outside of Arvy's World} % ganti judul soal

    \begin{tabular}{ | c c | }
        \hline
        Batas Waktu  & 1s \\    % jangan lupa ganti time limit
        Batas Memori & 128MB \\  % jangan lupa ganti memory limit
        \hline
    \end{tabular}
\end{center}

\subsection*{Deskripsi}

Arkavland pada tahun $2077$ sedang mengalami krisis ruang dan waktu, krisis ini mengakibatkan disrupsi waktu dan alur waktu yang tidak wajar.
Untuk menyelamatkan Arkavland, Arvy sedang meneliti dunia lain bernama Isokai untuk dihidupi penduduk Arkavland, dunia Isokai memiliki usia alam semesta $T$ satuan waktu.

Dalam penelitian ini, Arvy telah mengembangkan alat bernama TARKAVM \textit{(Time Anomaly Rescue Key And Vortex Manipulator)}. TARKAVM dapat digunakan untuk berpindah ke dunia Isokai pada waktu $t$, loncatan ini disebut \textit{Time Jump}.

Ternyata, anomali waktu dari dunia asal Arvy dapat mempengaruhi dunia Isokai pada waktu $X$. Anomali waktu ini akan membuat setiap penduduk Arkavland berpindah ke dunia asal.
Ketika penduduk Arkavland berada di dunia Isokai di waktu $X$, maka semua penduduk Arkavland akan terpindah ke dunia asal pada waktu $X+1$.
Ketika umur dunia Isokai telah habis, maka penduduk Arkavland juga akan terpindah ke dunia asal, kejadian ini juga dapat dibilang sebagai anomali waktu.

TARKAVM yang dikembangkan Arvy dapat melihat setiap anomali yang terjadi dalam usia dunia Isokai, TARKAVM juga dapat mendeteksi hilangnya anomali-anomali tersebut.
Arvy ingin penduduk Arkavland menghabiskan waktu selama mungkin di dunia Isokai,
maka setiap \textit{Time Jump} yang dilakukan TARKAVM akan memindahkan setiap penduduk Arkavland ke waktu $t$, sedemikian hingga $X-t+1$ adalah maksimal; $X$ adalah saat terjadinya anomali waktu terdekat dari $t$.
Dengan kata lain, \textit{Time Jump} yang dilakukan TARKAVM akan memaksimalkan waktu yang dimiliki penduduk Arkavland di dunia Isokai sebelum terpaksa pindah ke dunia asal.

Sayangnya TARKAVM belum sempurna, sehingga \textit{Time Jump} dapat terjadi kapan saja.

Diberikan Q data dari TARKAVM, jenis-jenis data adalah sebagai berikut:
\begin{itemize}
	\item1 t : Telah terjadi anomali waktu pada waktu t
	\item2 t : Anomali pada waktu t telah hilang, jika ada beberapa anomali waktu yang terjadi pada waktu t, maka semua anomali pada waktu tersebut hilang
	\item3 : TARKAVM melakukan Time Jump ke dunia Isokai sesuai dengan spesifikasi diatas
\end{itemize}
Untuk setiap data bertipe 3, tulislah waktu yang dihabiskan penduduk Arkavland di dunia Isokai.


\subsection*{Format Masukan}

Baris pertama terdiri dari dua bilangan bulat $T (1 \leq T \leq 10^1^8)$ dan $Q (1 \leq Q \leq 10^5)$. $Q$ baris berikutnya berisi data dari TARKAVM. Untuk setiap data \textbf{1 t} dan \textbf{2 t}, maka $1 \leq t \leq T$. Data \textbf{2 t} hanya dapat muncul jika sudah ada setidaknya satu anomali di waktu $t$.

\subsection*{Format Keluaran}

Keluaran berupa $N$ baris dengan $N$ merupakan banyaknya data tipe 3, tiap baris menyatakan durasi waktu yang dimiliki penduduk Arkavland setelah melakukan \textit{Time Jump} di dunia Isokai sebelum terpaksa pindah ke dunia asal.
\\

\begin{multicols}{2}
\subsection*{Contoh Masukan}
\begin{lstlisting}
6 6
3
1 3
1 5
3
2 5
3

\end{lstlisting}
\columnbreak
\subsection*{Contoh Keluaran}
\begin{lstlisting}
6
3
3
\end{lstlisting}
\vfill
\null
\end{multicols}

\subsection*{Penjelasan}
Umur dari dunia Isokai adalah 6 satuan waktu.

1 2 3 4 5 6


Data 1 mencatat bahwa TARKAVM melakukan \textit{Time Jump} ke $t = 1$, sehingga penduduk Arkavland memiliki 6 satuan waktu sebelum terpaksa pindah ke dunia asal.

\textbf{1 2 3 4 5 6}

Data 2 mencatat bahwa telah terjadi anomali pada waktu 3

1 2 3 | 4 5 6

Data 3 mencatat bahwa telah terjadi anomali pada waktu 5

1 2 3 | 4 5 | 6

Data 4 mencatat bahwa TARKAVM melakukan \textit{Time Jump} ke $t = 1$, \textit{Time Jump} yang dilakukan mengakibatkan penduduk Arkavland memiliki 3 satuan waktu sebelum terpaksa pindah ke dunia asal.

\textbf{1 2 3} | 4 5 | 6

Data 5 mencatat bahwa anomali pada waktu 5 telah hilang.

1 2 3 | 4 5 6

Data 6 mencatat bahwa TARKAVM melakukan \textit{Time Jump}, \textit{Time Jump} yang dilakukan mengakibatkan penduduk Arkavland memiliki 3 satuan waktu sebelum terpaksa pindah ke dunia asal.
\textit{Time Jump} yang dilakukan dapat dilakukan ke $t = 1$ atau $t = 4$.

\textbf{1 2 3} | 4 5 6
atau
1 2 3 | \textbf{4 5 6}

\textit{Catatan}: Boleh saja melakukan \textit{Time Jump} ke waktu yang sama berulang kali.

\pagebreak

\end{document}
