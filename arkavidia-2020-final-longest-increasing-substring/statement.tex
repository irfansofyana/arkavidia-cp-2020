\documentclass{article}

\usepackage{geometry}
\usepackage{amsmath}
\usepackage{graphicx}
\usepackage{listings}
\usepackage{hyperref}
\usepackage{multicol}
\usepackage{fancyhdr}
\pagestyle{fancy}
\hypersetup{ colorlinks=true, linkcolor=black, filecolor=magenta, urlcolor=cyan}
\geometry{ a4paper, total={170mm,257mm}, top=20mm, right=20mm, bottom=20mm, left=20mm}
\setlength{\parindent}{0pt}
\setlength{\parskip}{1em}
\renewcommand{\headrulewidth}{0pt}
\lhead{Competitive Programming - Arkavidia VI}
\fancyfoot[CE,CO]{\thepage}
\lstset{
    basicstyle=\ttfamily\small,
    columns=fixed,
    extendedchars=true,
    breaklines=true,
    tabsize=2,
    prebreak=\raisebox{0ex}[0ex][0ex]{\ensuremath{\hookleftarrow}},
    frame=none,
    showtabs=false,
    showspaces=false,
    showstringspaces=false,
    prebreak={},
    keywordstyle=\color[rgb]{0.627,0.126,0.941},
    commentstyle=\color[rgb]{0.133,0.545,0.133},
    stringstyle=\color[rgb]{01,0,0},
    captionpos=t,
    escapeinside={(\%}{\%)}
}

\begin{document}

\begin{center}
    \section*{Epidemi Penyakit} % ganti judul soal

    \begin{tabular}{ | c c | }
        \hline
        Batas Waktu  & 1s \\    % jangan lupa ganti time limit
        Batas Memori & 64MB \\  % jangan lupa ganti memory limit
        \hline
    \end{tabular}
\end{center}

\subsection*{Deskripsi}

Sebuah epidemi penyakit menyebar menjadi ancaman yang menyeramkan di seluruh dunia.
Badan kesehatan dunia telah mendata n kota yang terdampak oleh epidemi penyakit.
Untuk memudahkan pendataan, masing - masing kota diberi nama dari 1 hingga n. 
Untuk setiap kota , didata jumlah warganya yang tertular oleh epidemi penyakit tersebut.

Sebagai langkah antisipasi, Badan Kesehatan Dunia ingin mencari pola ketaraturan dari penyebaran epidemi.
Sebuah sequence kota dianggap memiliki kesamaan pola penyebaran dengan sequence kota lain jika 
setiap kota tersebut dapat di'rotasi' agar mendapat sequence kota yang lain. 
Sebagai upaya mengatasi epidemi, Badan kesehatan dunia ingin mencoba sebuah formulasi vaksin baru ke sejumlah kota.

Agar vaksin dapat berfungsi optimal, percobaan dilakukan kepada sebuah sequence kota yang memiliki kesamaan pola penyebaran dengan sequence kota lain yang jumlah warga yang tertular membentuk sequence yang menaik.

Anda sebagai programmer yang memiliki keinginan untuk membasmi epidemi tersebut, 
diminta untuk mencari berapa banyak kota maksimal yang dapat dilakukan uji percobaan vaksin !

\subsection*{Format Masukan}



\subsection*{Format Keluaran}


\\

\begin{multicols}{2}
\subsection*{Contoh Masukan}
\begin{lstlisting}

\end{lstlisting}
\columnbreak
\subsection*{Contoh Keluaran}
\begin{lstlisting}

\end{lstlisting}
\vfill
\null
\end{multicols}

\subsection*{Penjelasan}

\pagebreak

\end{document}
