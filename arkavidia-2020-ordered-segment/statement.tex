\documentclass{article}

\usepackage{geometry}
\usepackage{amsmath}
\usepackage{graphicx}
\usepackage{listings}
\usepackage{hyperref}
\usepackage{multicol}
\usepackage{fancyhdr}
\pagestyle{fancy}
\hypersetup{ colorlinks=true, linkcolor=black, filecolor=magenta, urlcolor=cyan}
\geometry{ a4paper, total={170mm,257mm}, top=20mm, right=20mm, bottom=20mm, left=20mm}
\setlength{\parindent}{0pt}
\setlength{\parskip}{1em}
\renewcommand{\headrulewidth}{0pt}
\lhead{Competitive Programming - Arkavidia V}
\fancyfoot[CE,CO]{\thepage}
\lstset{
    basicstyle=\ttfamily\small,
    columns=fixed,
    extendedchars=true,
    breaklines=true,
    tabsize=2,
    prebreak=\raisebox{0ex}[0ex][0ex]{\ensuremath{\hookleftarrow}},
    frame=none,
    showtabs=false,
    showspaces=false,
    showstringspaces=false,
    prebreak={},
    keywordstyle=\color[rgb]{0.627,0.126,0.941},
    commentstyle=\color[rgb]{0.133,0.545,0.133},
    stringstyle=\color[rgb]{01,0,0},
    captionpos=t,
    escapeinside={(\%}{\%)}
}

\begin{document}

\begin{center}
    \section*{Ordered Segment} % ganti judul soal

    \begin{tabular}{ | c c | }
        \hline
        Batas Waktu  & 1s \\    % jangan lupa ganti time limit
        Batas Memori & 32MB \\  % jangan lupa ganti memory limit
        \hline
    \end{tabular}
\end{center}

\subsection*{Deskripsi}

Elza adalah remaja berumur 18 tahun yang rajin dan hobi membaca buku.
Hobinya tersebut didapat dari ayahnya yang juga hobi membaca buku.
Waktu kecil ayahnya selalu membacakannya dongeng hingga Elza kecil tertidur.
Momen indah itu selalu Elza ingat saat Elza senang maupun sedih.
Kini Elza sudah menjadi mahasiswa.
Kesibukan mengerjakan tugas dan belajar tidak mengahalangi Elza untuk membaca buku.
Elza sangat senang membaca buku.

Cahaya matahari keoranyean menyinari kamar Elza dan menemaninya terhanyut dalam buku yang ia baca.
Lemari berisi buku-buku sebanyak $N$ memantulkan keindahan cahaya keoranyean itu.
Buku-buku berdiri secara acak karena Elza belum sempat membereskan.
Diantara buku-buku terdapat judul buku kesukaan Elza yang ayahnya sering bacakan kepada Elza saat kecil.
Buku-buku bervolume kesukaan Elza tersebut dibatasi oleh indeks $L$ dan $R$ sebanyak $Q$ bagian.
Namun apakah setiap bagian yang dibatasi $L$ dan $R$ terurut menaik?

\subsection*{Format Masukan}

Pada baris pertama dimasukkan banyaknya buku.
Pada baris kedua dimasukkan volume buku.
Pada baris ketiga dimasukkan banyaknya bagian.
pada baris keempat dimasukkan batasan bagian.

\subsection*{Format Keluaran}

Keluarkan erurut atau tidaknya bagian yang dipisah oleh \textit{newline}.
\\

\begin{multicols}{2}
\subsection*{Contoh Masukan}
\begin{lstlisting}
7
2 4 5 2 1 2 3
3
1 3
2 5
6 6
\end{lstlisting}
\columnbreak
\subsection*{Contoh Keluaran}
\begin{lstlisting}
Terurut
Tidak Terurut
Terurut
\end{lstlisting}
\vfill
\null
\end{multicols}

% \subsection*{Penjelasan}
% Jika dibutuhkan, tambahkan penjelasan di sini

\pagebreak

\end{document}
